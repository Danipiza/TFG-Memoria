\chapter{Diseño e Implementaciones}
	
	El objetivo de este proyecto es mejorar los algoritmos de IA mencionados anteriormente. En este capítulo se presentan los diseños e implementaciones desarrolladas a lo largo del desarrollo del mismo.
	
	\newpage %TODO QUITAR
	\section{Ejemplos básicos MPI}
	
	
	\newpage %TODO QUITAR
	\section{Algoritmos de Clustering}
	
		\subsection{Jerárquico Aglomerativo}
			Este algoritmo usa una matriz para calcular las agrupaciones. Como es una matriz simétrica, podemos reducir la complejidad espacial usando solo el triángulo superior.
			
		\subsection{K-Medias}
			De las técnicas de clustering de aprendizaje no supervisado, en la cual tenemos una población inicial de individuos sin clasificar, y un valor K sujeto a una asignación flexible según nuestros criterios. Al contrario al algoritmo anterior no se usa una matriz, y solo se usa distancia por centroides.
			
		\subsection{K-Vecinos más cercanos (KNN)}
			Tenemos un valor K  asignado de manera arbitraria como en el algoritmo de K-Medias. Esta técnica de clustering pertenece al aprendizaje supervisado, tenemos una población de individuos categorizados con las etiquetas de asignacion de cluster. Una población a predecir.
			
	
	
	\newpage %TODO QUITAR
	\section{Aprendizaje por refuerzo}
		El algoritmo de Q-Learing actualiza iterativamente las estimaciones de calidad de las acciones permitidas en el entorno de desarrollo. Estos valores se almacenan en la Q-Table, representado como una matriz en la que cada fila es un estado, y las columnas son las acciones disponibles.
		
	
	\newpage %TODO QUITAR
	\section{Algoritmos Evolutivos}
		Los algoritmos evolutivos son sencillos de paralelizar, debido a que son procesos que se repiten muchas veces, y se ejecutan en muchos individuos.
	
	\newpage %TODO QUITAR
	\section{Redes Neuronales}
		Esta poderosa herramienta de aprendizaje supervisado, está diseñada para reconocer patrones complejos y realizar diversas tareas. Aprende con un proceso iterativo de entrenamiento, ajustando las conexiones entre neuronas. Este proceso secuencial es complejo de paralelizar. Al finalizar una predicción el modelo se tiene que actualizar propagando hacia atrás.
	
