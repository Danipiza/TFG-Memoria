\chapter{Introducción}

	En este capítulo se presenta una perspectiva general del contexto en el que se ha llevado a cabo el proyecto. Además de los desafíos enfrentados durante su desarrollo para alcanzar las contribuciones mencionadas, se detallan cada uno de los propósitos perseguidos en él.
	

	\section{Introduccion LaTeX TODO QUITAR}
		The document is divided into \texttt{chapters}, \texttt{sections}, and \texttt{subsections}.

		Some important references are \cite{einstein,latexcompanion,knuthwebsite}.

		To add paragraphs in the document, 
		one line break is not enough,

		two line breaks are needed.

		An itemized list:

		\begin{itemize}
			\item An item.
			\item Another item.
			\item Final item.
		\end{itemize}

		An enumerated list:

		\begin{enumerate}
			\item First item.
			\item Second item.
			\item Third item.
		\end{enumerate}

		A figure with an image is presented in \Cref{fig:figura}. Note that it floats away and latex places it where convenient.

		\begin{figure}
			\centering
			\includegraphics[width=0.4\textwidth]{Images/escudo_ucm.pdf}
			\caption{Sample figure}
			\label{fig:figura}
		\end{figure}

		Tables work in the same way, as seen in \Cref{tab:tabla}

		\begin{table}
			\centering
			\begin{tabular}{c|c|c}
				Row & English & Español \\\hline\hline
				1 & One & Uno \\
				2 & Two & Dos \\
			\end{tabular}
			\caption{Sample table}
			\label{tab:tabla}
		\end{table}
	\blindtext
	
	\newpage %TODO QUITAR
	\section{Definición y alcance del proyecto}
		El desarrollo de las Inteligencias Artificiales en nuestra sociedad ha sido un fenómeno de gran relevancia, además de popular, en los últimos años. Estas tecnologías han llegado para quedarse. Están mejorando nuestra calidad de vida, desde la automatización de tareas hasta la asistencia virtual, estas IAs desempeñan un papel cada vez más importante en nuestro día a día.
		Con el advenimiento del Internet de alta velocidad y la proliferación de datos, las empresas tecnológicas se enfrentan a la necesidad creciente de desarrollar servicios de alta calidad en un mercado muy competitivo. Se invierte mucho dinero y tiempo en mejorar y diseñar algoritmos, para implementar Inteligencias Artificiales para el acceso público
		
		TODO...
	
	\newpage  %TODO QUITAR
	\section{Motivación}
		Actualmente hay muchas implementaciones de algoritmos de IA. Scikit learn es una biblioteca de python perfecta para probar cualquier técnica. Secuencialmente está perfeccionado y demuestra un alto desempeño computacional, pero tiene sus limitaciones. 
		
		TODO...
		
	
	\newpage  %TODO QUITAR
	\section{Objetivo}
		El objetivo principal de este trabajo es paralelizar varios algoritmos de IA, desarrollando varias implementaciones que reduzcan el tiempo de ejecución. Además de evaluar dichas mejoras para optimizarlas lo máximo posible.	
		
	\newpage  %TODO QUITAR
	\section{Estructura del documento}
		El resto de este documento está organizado en los siguientes capítulos:
		
		
		\begin{itemize}
			\item Capítulo 2, Contextualización. Proporcionar información de cada algoritmo estudiado, para la correcta lectura del trabajo.
			
			\item Capítulo 3, Diseño e implementaciones. Comenzando con unos ejemplos básicos fuera del ámbito de la inteligencia artificial, seguido de las mejoras desarrolladas para las diferentes técnicas abordadas.
			
			\item Capítulo 4, Estudio empírico. Presenta el estudio realizado, el cual consiste en medir los tiempos de las mejoras así como las implementaciones secuenciales, para poder medir el speed-up y realizar comparaciones significativas.
			
			\item Capítulo 5, Conclusiones y trabajo a futuro.
		\end{itemize}
		
	
		
		
		
		
		
		
		
		
		
	