\chapter{Contextualización}
	En este capítulo se presenta una breve descripción de los algoritmos de Inteligencia Artificial que se van a profundizar a lo largo del proyecto. Así como los usos y características. 
	
	El objetivo de este capítulo es explicar rápidamente los algoritmos, y facilitar la lectura de los capítulos posteriores. Que desarrollen las mejoras realizadas y resultados obtenidos.
	
	
	\newpage %TODO QUITAR
	\section{MPI}
		Message Passing Interface[8]  (MPI)  es un estándar para una biblioteca de paso de mensajes, diseñado para funcionar en una amplia variedad de arquitecturas informáticas paralelas. Permite la comunicación entre procesos, mandando y recibiendo mensajes de todo tipo. Comúnmente usado en informática de alto rendimiento[9] (HPC) y entornos informáticos paralelos, para desarrollar aplicaciones paralelas escalables y eficientes.
		
		TODO...
	
	\newpage	
	\section{Aprendizaje por Refuerzo}
		Reinforcement Learning (RL), en español Aprendizaje por Refuerzo. Es un tipo de aprendizaje automático donde el agente aprende en base a las decisiones tomadas al interactuar con el entorno. El agente aprende a llegar a una meta o maximizar un cúmulo de recompensas obtenidas al realizar un determinado número de acciones consecutivas, y observar las posibles recompensas al realizar cada acción en los estados del entorno.
		
		TODO...
		
	\newpage %TODO QUITAR
	\section{Aprendizaje No-Supervisado}
		Los métodos no supervisados (unsupervised methods) son algoritmos de aprendizaje automático que basan su proceso en un entrenamiento con datos sin etiquetar. Es decir, a priori no se conoce ningún valor objetivo, ya sea categórico o numérico. 
		
		TODO...
	
	\newpage %TODO QUITAR
	\section{Aprendizaje Supervisado}
		Al contrario que el apartado anterior, este tipo de aprendizaje automático, es entrenado con un dataset categorizado con su salida correcta. El algoritmo aprende de este conjunto, para hacer predicciones sobre unos datos desconocidos.
		
		
		TODO...
			
	\newpage %TODO QUITAR
	\section{Algoritmos Evolutiva}
		La programación evolutiva es una técnica de optimización inspirada en la teoría de la evolución biológica. Se basa en el concepto de selección natural y evolución de las poblaciones para encontrar soluciones a problemas complejos. 
		
	
		TODO...
	
	