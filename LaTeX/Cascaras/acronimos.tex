% Fichero con acr�nimos para su uso con GlossTeX
%
% El listado de acr�nimos funciona de manera similar a las referencias
% bibliogr�ficas. Este  fichero es equivalente por tanto  al .bib, con
% el listado de  los acr�nimos disponibles. A lo  largo del documento,
% cuando se  utilice alguno  de ellos, se  referencia para  indicar al
% sistema que queremos que dicho  acr�nimo aparezca en el glosario. En
% contra de  lo que ocurre con  la bibliograf�a, el  texto original no
% sufre ninguna modificaci�n.
%
% Las entradas son:
% @entry{<etiqueta>[,<item>[,<forma larga>]]} [<texto>]
%
% La etiqueta es el  nombre del acr�nimo (equivalente a la etiqueta en
% BibTeX) que luego se utiliza para "referenciarlo" en el texto.

% Para  que  la  generaci�n   en  Release  con  el  Makefile  funcione
% correctamente, necesitar�s  al menos tener  referenciado un acr�nimo
% dentro del  texto.  Si  no quieres usar  acr�nimos, o de  momento no
% tienes ninguno  referenciado, basta con que no  definas la constante
% \acronimosEnRelease en config.tex





\begin{flushleft}
	\huge{\textbf{Acrónimos}}
\end{flushleft}

\vspace{1cm}

\textbf{MPAI} Message Passing Artificial Inteligence
\newacronym{MPAI}{MPAI}{Message Passing Artificial Inteligence}
\vspace{0.5cm}

\textbf{AI} Artificial Intelligence
\newacronym{AI}{AI}{Artificial Intelligence}
\vspace{0.5cm}

\textbf{MPI} Message Passing Interface 
\newacronym{MPI}{MPI}{Message Passing Interface}
\vspace{0.5cm}

\textbf{CPU} Central Processing Unit
\newacronym{CPU}{CPU}{Central Processing Unit}
\vspace{0.5cm}

\textbf{GB} Giga-Byte 
\newacronym{GB}{GB}{Giga-Byte}
\vspace{0.5cm}

\textbf{RAM} Random Access Memory
\newacronym{RAM}{RAM}{Random Access Memory}
\vspace{0.5cm}

\textbf{CO2} Carbon Dioxide
\newacronym{CO2}{CO2}{Carbon Dioxide}
\vspace{0.5cm}

\textbf{HPC} High Performance Computing
\newacronym{HPC}{HPC}{High Performance Computing}
\vspace{0.5cm}

\textbf{SPMD} Single Program Multiple Data
\newacronym{SPMD}{SPMD}{Single Program Multiple Data}
\vspace{0.5cm}

\textbf{RL} Reinforcement Learning
\newacronym{RL}{RL}{Reinforcement Learning}
\vspace{0.5cm}

\textbf{MDP} Markov Decision Process
\newacronym{MDP}{MDP}{Markov Decision Process}
\vspace{0.5cm}

\textbf{DQN} Deep Q-Network
\newacronym{DQN}{DQN}{Deep Q-Network}
\vspace{0.5cm}

\textbf{KNN} K-Nearest Neighbors
\newacronym{KNN}{KNN}{K-Nearest Neighbors}
\vspace{0.5cm}

\textbf{PEV} Programación EVolutiva
\newacronym{PEV}{PEV}{Programación EVolutiva}
\vspace{0.5cm}


