% +--------------------------------------------------------------------+
% | Copyright Page
% +--------------------------------------------------------------------+

\newpage

\thispagestyle{empty}

\begin{center}

{\bf \Huge Resumen}

  \end{center}
\vspace{1cm}

El trabajo que se presenta se enfoca en la optimización de algoritmos de Inteligencia  Artificial (IA) mediante el uso de MPI (Message Passing Interface), una biblioteca estándar desarrollada para el cómputo de alto rendimiento. 
El objetivo principal consiste en reducir el tiempo de ejecución de los algoritmos explotando el paralelismo de los recursos de cómputo y la memoria distribuida. Esta tarea es especialmente relevante debido al alto coste computacional y de recursos que implica entrenar o ejecutar estos algoritmos.

Este proyecto incluye una descripción de los fundamentos teóricos de los algoritmos que se van a implementar, así como el funcionamiento de la biblioteca MPI. 
Una vez puesto en contexto, se desarrollan en profundidad las estrategias propuestas para mejorar los algoritmos.
Además, se ha realizado un estudio empírico para analizar las mejoras desarrolladas a lo largo del proyecto. Este estudio incluye la ejecución de las mejoras en un sistema distribuido que consta de 128 núcleos de CPU y 256 GB de RAM.



\vspace{1cm}

% +--------------------------------------------------------------------+
% | On the line below, repla	ce Fecha
% |
% +--------------------------------------------------------------------+

\begin{center}

{\bf \Large Palabras clave}

   \end{center}

   \vspace{0.5cm}
   
   IA, aprendizaje automático, MPI, speedup, memoria distribuida, redes neuronales, algoritmos evolutivos, clustering
   
   


