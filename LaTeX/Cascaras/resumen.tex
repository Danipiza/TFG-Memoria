% +--------------------------------------------------------------------+
% | Copyright Page
% +--------------------------------------------------------------------+

\newpage

\thispagestyle{empty}

\begin{center}

{\bf \Huge Resumen}

  \end{center}
\vspace{1cm}

El trabajo que se presenta, se enfoca en la optimización de algoritmos de Inteligencia  Artificial (IA). Mediante el uso de MPI (Message Passing Interface), una biblioteca estándar desarrollada para el cómputo de alto rendimiento. 
El objetivo principal, reducir el tiempo de ejecución de algoritmos de IA, mediante paralelización usando memoria distribuida. Es una tarea muy importante debido al alto costo temporal y de recursos que implica entrenar o ejecutar estos modelos.

Comenzamos con una explicación de los fundamentos teóricos de los algoritmos que se van a implementar, así como explicar el funcionamiento de la biblioteca MPI. 
Una vez puesto en contexto, se desarrollan en profundidad las estrategias implementadas para mejorar los algoritmos.
Y para finalizar se ha realizado una fase de experimentación para analizar las mejoras desarrolladas a lo largo del proyecto. Esta evaluación incluye la ejecución de las mejoras en un superordenador con 128 núcleos.

\vspace{1cm}

% +--------------------------------------------------------------------+
% | On the line below, repla	ce Fecha
% |
% +--------------------------------------------------------------------+

\begin{center}

{\bf \Large Palabras clave}

   \end{center}

   \vspace{0.5cm}
   
   IA, aprendizaje automático, MPI, speedup, memoria distribuida, redes neuronales, algoritmos evolutivos, clustering
   
   


