\chapter{Conclusiones y trabajo futuro}
\label{cap:c5_conclu}

	En este trabajo se han desarrollado varias mejoras en distintos algoritmos de IA, a través de la biblioteca estándar de paso de mensajes MPI. Los desafíos encontrados durante su desarrollo resultaron ser más complejos de lo que se había previsto inicialmente. Los problemas de configuración de la biblioteca MPI en windows, y adaptar la gestión de bibliotecas de Python usando Anaconda, fueron unos problemas completamente imprevistos. El desconocimiento general de  MPI, se debe a la ausencia de asignaturas específicas de programación distribuidas en el grado de Ingeniería Informática, únicamente ofreciendo fundamentos teóricos, sin profundizar en la práctica. La ejecución de las pruebas en el sistema altamente distribuido de la Facultad de Informática, junto con la documentación del trabajo desarrollado, tomó más tiempo del calculado. Sin embargo fue una étapa sumamente gratificante e interesante. 
	
	
	Una vez finalizadas las estrategias, se ha llevado a cabo una fase de experimentación. Consistiendo en analizar los tiempos de ejecución, variando todos los parámetros disponibles, además de variar los conjuntos de poblaciones para cada tipo de algoritmo. Las ejecuciones de las pruebas requieren un coste computacional alto, además de mucho tiempo para finalizar. Se puede constatar que, aunque algunas estrategias no funcionaron como se esperaba, es posible extraer conclusiones valiosas del trabajo implementado.
	
	Una de las cosas más importantes que he aprendido a lo largo del trabajo es que no siempre más es mejor. Utilizar más procesos no tiene porque derivar en un rendimiento proporcional al trabajo ejecutado. La sobrecarga (overhead) de los procesos en las implementaciones es un fundamento a tener en cuenta a la hora de ejecutar programas, y en la vida misma.	
	
	
	Como trabajo a futuro se propone investigar otros algoritmos de las técnicas desarrolladas. Además de investigar y mejorar otras técnicas de IA, como puede ser el procesamiento del lenguaje natural.
	
	
	
	
	