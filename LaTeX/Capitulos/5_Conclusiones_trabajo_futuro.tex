\chapter{Conclusiones y trabajo futuro}
\label{cap:c5_conclu}

	En este trabajo se han desarrollado varias mejoras en distintos algoritmos de IA, a través de la biblioteca estándar de paso de mensajes MPI. Los desafíos encontrados durante su desarrollo resultaron ser más complejos de lo que se había previsto inicialmente. Los problemas de configuración de la biblioteca MPI en windows, así como y adaptar la gestión de bibliotecas de Python usando Anaconda, fueron unos problemas completamente imprevistos. El desconocimiento general de  MPI se debe a la ausencia de asignaturas específicas de programación distribuidas en el grado de Ingeniería Informática, únicamente ofreciendo fundamentos teóricos, sin profundizar en la práctica. La ejecución de las pruebas en el sistema distribuido de la Facultad de Informática, junto con la documentación del trabajo desarrollado, tomó más tiempo del calculado. Sin embargo, fue una étapa sumamente gratificante e interesante. 
	
		
	Una vez finalizadas las estrategias propuestas, se ha llevado a cabo una fase de experimentación, que ha consistido en analizar los tiempos de ejecución, variando tanto los parámetros disponibles, como los conjuntos de poblaciones para cada tipo de algoritmo. Cabe destacar el alto coste computacional de los experimentos. Por ejemplo, en el algoritmo jerárquico aglomerativo con distancia por enlace simple, una prueba en el sistema  distribuido requirió más de un día en finalizar, obligando a reducir el tamaño de la población usada. Es importante remarcar que algunos de los resultados obtenidos no coincidían con la tendencia esperada. En particular, la estrategia de segmentación realizada en las redes neuronales, donde el rendimiento alcanzado fue peor que el algoritmo secuencial a pesar de contar con -al menos- tres procesos. Aparte de disponer con los resultados obtenidos en la misma estrategia pero para los algoritmos evolutivos, en los cuales si se obtuvieron buenos resultados. Después de realizar un análisis, observamos que la causa de estos resultados se debe a contar con dos flujos de mensajes en direcciones opuestas. La importancia de los buenos resultados, es equivalente a obtener resultados no tan eficaces como se esperaban, pues es un avance para extraer conclusiones u otras ideas a implementar.
	

	Una de las cosas más importantes que he aprendido a lo largo del trabajo es que no siempre \textit{más es mejor}. Utilizar más procesos no tiene por qué derivar en un rendimiento proporcional al trabajo ejecutado. La sobrecarga (\textit{overhead}) de los procesos en las implementaciones es un fundamento a tener en cuenta a la hora de ejecutar programas, y en la vida misma.	
	
	
	Como trabajo a futuro se propone investigar otros algoritmos de las técnicas desarrolladas, además de investigar y mejorar otras técnicas de IA, como puede ser el procesamiento del lenguaje natural.
	
	
	
	
	