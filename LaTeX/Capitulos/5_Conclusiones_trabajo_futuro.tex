\chapter{Conclusiones y trabajo futuro}
\label{cap:c5_conclu}

	En este trabajo se han desarrollado varias mejoras en distintos algoritmos de IA, a través de la biblioteca estándar de paso de mensajes MPI.
	Los desafíos encontrados durante su desarrollo resultaron ser más complejos de lo que se había previsto inicialmente. Los problemas de configuración de la biblioteca MPI en windows, y adaptar la gestión de bibliotecas de Python usando Anaconda, fueron unos problemas completamente imprevistos. El desconocimiento general de  MPI, se debe a la ausencia de asignaturas específicas de programación distribuidas en el grado de Ingeniería Informática, únicamente ofreciendo fundamentos teóricos, sin profundizar en la práctica. 
	La escasa implementación práctica en las asignaturas de IA, en el itinerario Tecnología Específica de Computación del tercer curso ha derivado en tener que invertir más tiempo en investigar e implementar los algoritmos, proceso que he encontrado satisfactorio. La teoría vista en clase fue muy útil para el desarrollo del trabajo, pero usar exclusivamente la librería sklearn, de scikit-learn, provocó un desconocimiento de código para implementar estos algoritmos.
	
	Una vez finalizadas las implementaciones, se ha llevado a cabo una fase de experimentación. Consistiendo en analizar los tiempos de ejecución, variando todos los parámetros disponibles, además de variar los conjuntos de poblaciones para cada tipo de algoritmo.
	Las ejecuciones de las pruebas requieren un coste computacional alto, además de mucho tiempo para finalizar. Para pruebas pequeñas no se consiguen apreciar reducciones significativas. Normalmente se pierde tiempo al paralelizar. Pero conforme aumentan los parámetros introducidos, mejora notablemente el speedup de las mejoras.
	
	Cabe destacar que no siempre utilizar más procesos deriva en un mejor rendimiento, la sobrecarga de los procesos en las implementaciones es un fundamento a tener en cuenta a la hora de ejecutar programas.	
	
	
	Como trabajo a futuro se propone investigar otros algoritmos de las técnicas desarrolladas. Además de investigar y mejorar otras técnicas de IA, como puede ser el procesamiento del lenguaje natural.
	
	
	
	
	